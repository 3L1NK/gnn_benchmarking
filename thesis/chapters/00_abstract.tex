This thesis benchmarks graph-based and non-graph models for next-day multi-asset return prediction under one fixed and reproducible protocol. 
The evaluation spans the out-of-sample period from January 2, 2020 to December 27, 2024 and compares XGBoost, LSTM, and multiple graph neural network (GNN) variants under daily and weekly rebalancing.

The results show three robust patterns. First, prediction error metrics (RMSE/MAE) are tightly clustered across models, so point-error gains are modest. Second, risk-adjusted portfolio outcomes are materially better under weekly rebalancing, with much lower turnover than daily trading. Third, the best overall model is the graph-feature baseline \texttt{xgb\_node2vec\_corr\_tuned\_all}, which reaches annualized Sharpe 0.806 and final value 2.011 at rebalance frequency 5.

At the model-family level, XGBoost variants achieve the highest average Sharpe under both rebalancing policies, while static GNN variants remain competitive but less consistent. Edge ablations indicate that sector-based graphs are strongest on average for GNNs in the tuned experiment set; combined edge graphs do not consistently improve outcomes.

Despite model-level improvements, all tested model-driven strategies underperform the test-window buy-and-hold baseline (final value 5.049). The thesis therefore concludes that relational structure is useful, but practical deployment requires stronger robustness checks, explicit transaction-cost sensitivity analysis, and strict time-aware graph construction controls to exclude leakage risk.
