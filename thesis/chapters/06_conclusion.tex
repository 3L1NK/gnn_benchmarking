This thesis evaluated graph-aware and non-graph models for next-day multi-asset forecasting under one fixed protocol with reproducible artifact generation.

\section{Direct Answers to Research Questions}
\textbf{RQ1: Do graph-based approaches improve predictive quality?}
Partially. Graph-aware information is valuable, but the strongest result is a graph-feature hybrid (Node2Vec + XGBoost) rather than a universal end-to-end GNN winner.

\textbf{RQ2: Which edge signals are most reliable in this benchmark?}
Sector-based edges are the most stable average performer for GNN variants. Correlation-only and merged-edge designs can work in selected runs but are less robust on average.

\textbf{RQ3: Do prediction gains translate into portfolio gains?}
Partially. Relative improvements in risk-adjusted metrics are present across learned models, yet absolute final wealth remains below buy-and-hold in this test regime.

\textbf{RQ4: How sensitive are conclusions to rebalance policy?}
Highly sensitive. Weekly rebalancing consistently lowers turnover and typically improves annualized Sharpe versus daily trading.

\section{Strongest Supported Claim}
The strongest defensible claim is:
\begin{quote}
Under a fixed leakage-aware protocol and explicit transaction costs, relational information improves model selection quality, but implementation policy (especially rebalancing frequency) is as important as architecture choice for deployable performance.
\end{quote}

\section{What Was Fixed in the Thesis Production Stack}
The final pipeline now enforces:
\begin{itemize}
  \item stable run identity (\texttt{run\_key}, \texttt{split\_id}, \texttt{config\_hash}),
  \item ledger-level deduplication (latest run wins),
  \item fail-fast metric checks (no empty/constant return series),
  \item canonical report generation from tuned-all ledger,
  \item run-matrix size and plot-point sanity checks,
  \item bibliography quality checks blocking placeholder references.
\end{itemize}
These controls directly address reviewer-risk issues (NaNs, duplicates, one-point plots, and citation placeholders).

\section{Limitations}
The conclusions are bounded by:
\begin{itemize}
  \item one US-focused universe and one dominant test regime (2020--2024),
  \item a single long-only top-$k$ portfolio mapping,
  \item limited seed expansion relative to full uncertainty quantification,
  \item mostly static graph construction assumptions.
\end{itemize}

\section{Future Work}
High-priority extensions are:
\begin{enumerate}
  \item transaction-cost sensitivity sweeps (0/5/10+ bps) across the full matrix,
  \item multi-seed and block-bootstrap confidence intervals for rank stability,
  \item rolling-window and regime-segmented evaluation,
  \item dynamic graph refresh policies with stricter time-index validation,
  \item alternative portfolio mappings (risk-budgeted, market-neutral, constrained long-short).
\end{enumerate}

\section{Final Statement}
The evidence supports a restrained conclusion: graph structure is useful, but claims must remain protocol-conditional and portfolio-aware. Robust thesis-grade benchmarking requires not only model experimentation, but also strong reporting integrity, reproducibility contracts, and citation quality discipline.
