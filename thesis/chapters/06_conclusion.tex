This thesis examined whether graph-aware modeling improves next-day multi-asset forecasting and downstream portfolio outcomes under one reproducible benchmark protocol. The central result is nuanced: relational information is clearly useful, but the strongest performer is a graph-feature hybrid (\texttt{xgb\_node2vec\_corr}) rather than a universal end-to-end GNN winner.

\section{Answers to the Research Questions}
\textbf{RQ1: Do graph-based approaches improve predictive quality?}  
Partially. End-to-end GNNs do not dominate all criteria, but graph-aware representations improve decision quality in competitive variants and materially contribute to the top hybrid model.

\textbf{RQ2: Which edge signals are most reliable in this benchmark?}  
Sector-driven edges produce the strongest average GNN outcomes across rebalance policies. Correlation-only and mixed-edge variants can be competitive in specific runs, but they are less consistently robust in aggregate.

\textbf{RQ3: Do predictive gains translate into portfolio gains?}  
Yes, but with constraints. Ranking quality gains can improve risk-adjusted metrics, yet absolute wealth outcomes remain below the buy-and-hold benchmark in the 2020--2024 test regime.

\section{Main Empirical Conclusions}
\begin{itemize}
  \item Weekly rebalancing (\texttt{rebalance\_freq=5}) consistently improves annualized Sharpe and lowers turnover versus daily rebalancing.
  \item The top run (\texttt{xgb\_node2vec\_corr\_tuned\_all}, reb5) reaches annualized Sharpe 0.806 with final value 2.011.
  \item On family averages, XGBoost variants lead risk-adjusted performance, while GNNs remain competitive but not dominant.
  \item Edge ablation indicates that added graph complexity does not automatically improve deployment quality.
\end{itemize}

\section{Methodological Contributions}
Beyond model comparison, this work contributes an operational thesis framework:
\begin{itemize}
  \item a single leakage-aware split and target-policy contract,
  \item an artifact-first experimental ledger for reproducible reporting,
  \item deterministic decision ranking aligned with deployable constraints,
  \item direct synchronization between experiment outputs and LaTeX tables/figures.
\end{itemize}
This structure supports transparent reruns and reduces manual reporting error.

\section{Practical Interpretation}
From a deployment perspective, the benchmark supports a pragmatic strategy: use graph signal where it is robust, but avoid architecture complexity that is not justified by incremental risk-adjusted gains. In this dataset and regime, hybrid graph-feature pipelines currently offer the strongest balance between performance and operational stability.

\section{Limitations}
Several limitations bound external validity:
\begin{itemize}
  \item single historical test regime (2020--2024) with strong macro and sector rotations,
  \item fixed long-only top-$k$ execution that may mask alpha in alternative constructions,
  \item incomplete uncertainty quantification from limited seed expansion on neural variants,
  \item static edge definitions that may underrepresent rapidly changing cross-asset links.
\end{itemize}

\section{Future Work}
High-priority follow-up work includes:
\begin{itemize}
  \item transaction-cost sensitivity sweeps (e.g., 0/5/10 bps) across all models,
  \item multi-seed and block-bootstrap confidence intervals for ranking stability,
  \item dynamic graph refresh policies and stricter time-indexed edge validation,
  \item market-neutral long-short and risk-budgeted portfolio overlays,
  \item broader cross-market validation to test transferability.
\end{itemize}

\section{Final Statement}
The evidence supports a restrained conclusion: graph structure is valuable for equity forecasting, but value is context-dependent and best realized through disciplined benchmarking, strict causal evaluation, and robust portfolio-aware selection criteria.
