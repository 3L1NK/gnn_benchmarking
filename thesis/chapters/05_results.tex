This chapter reports benchmark outcomes generated from \texttt{results/results\_tuned\_all.jsonl} through the hardened reporting pipeline. All tables and figures are synchronized from \texttt{results/reports/thesis\_tuned\_all/}. Every summary in this chapter is derived from the same canonical deduplicated dataframe (\texttt{master\_comparison.csv}) and checked against the canonical run matrix.

\section{Evaluation Context and Baselines}
The out-of-sample test window is January~2,~2020 to December~27,~2024. Baseline context is exported explicitly to avoid horizon mismatch between model equity and benchmark equity.
% Auto-generated by thesis/scripts/export_tables.py
\begin{table}[htbp]
\caption{Buy-and-hold baseline context used for report-relative interpretation.}
\label{tab:baseline-context}
\centering
\small
\setlength{\tabcolsep}{4pt}
\resizebox{\textwidth}{!}{%
\begin{tabular}{lllrllr}
\toprule
Reb & Global Start & Global End & Global Final & Test Start & Test End & Test Final \\
\midrule
reb=1 & 2000-03-29 & 2024-12-30 & 51.1784 & 2020-01-02 & 2024-12-27 & 5.0493 \\
reb=5 & 2000-03-29 & 2024-12-30 & 51.1784 & 2020-01-02 & 2024-12-27 & 5.0493 \\
\bottomrule
\end{tabular}
}
\end{table}


The global buy-and-hold curve (from 2000 onward) is substantially higher than the rebased test-window curve by construction, so all comparative statements in this chapter use the test-window baseline only.

\section{Runtime and Compute Profile}
% Auto-generated by thesis/scripts/export_tables.py
\begin{table}[htbp]
\caption{Runtime summary by model family and rebalance policy.}
\label{tab:runtime-summary}
\centering
\small
\setlength{\tabcolsep}{4pt}
\resizebox{\textwidth}{!}{%
\begin{tabular}{lrrrrrrrrr}
\toprule
Family & Reb & Train Mean (s) & Train Median (s) & Train Min (s) & Train Max (s) & Infer Mean (s) & Infer Median (s) & Infer Min (s) & Infer Max (s) \\
\midrule
GNN & 1 & 14131.9029 & 58.5151 & 22.1728 & 139818.8415 & 21.0830 & 20.8247 & 20.4412 & 22.6119 \\
LSTM & 1 & 254.4480 & 254.4480 & 254.4480 & 254.4480 & 9.5007 & 9.5007 & 9.5007 & 9.5007 \\
XGB & 1 & 40.8929 & 40.8929 & 1.2379 & 80.5480 & 0.0547 & 0.0547 & 0.0213 & 0.0881 \\
GNN & 5 & 14131.9029 & 58.5151 & 22.1728 & 139818.8415 & 21.0830 & 20.8247 & 20.4412 & 22.6119 \\
LSTM & 5 & 254.4480 & 254.4480 & 254.4480 & 254.4480 & 9.5007 & 9.5007 & 9.5007 & 9.5007 \\
XGB & 5 & 40.8929 & 40.8929 & 1.2379 & 80.5480 & 0.0547 & 0.0547 & 0.0213 & 0.0881 \\
\bottomrule
\end{tabular}
}
\end{table}


Runtime diagnostics show a strong family-level separation: graph models spend most wall time in training and graph-side preprocessing, while XGBoost variants train quickly and infer almost instantly. This is relevant for deployment engineering, where retraining cadence can dominate architecture choice.

\section{Visual Diagnostics by Rebalance Policy}

\subsection{Equity Trajectories}
\begin{figure}[H]
  \centering
  \includegraphics[width=\textwidth]{equity_curves_key_models_reb1.png}
  \caption{Equity curves for key model variants under daily rebalancing (\texttt{rebalance\_freq=1}).}
  \label{fig:equity-curves-reb1}
\end{figure}

\begin{figure}[H]
  \centering
  \includegraphics[width=\textwidth]{equity_curves_key_models_reb5.png}
  \caption{Equity curves for key model variants under weekly rebalancing (\texttt{rebalance\_freq=5}).}
  \label{fig:equity-curves-reb5}
\end{figure}

\subsection{Risk-Adjusted and Prediction Trade-Offs}
\begin{figure}[H]
  \centering
  \includegraphics[width=\textwidth]{ic_vs_sharpe_reb1.png}
  \caption{Rank IC versus annualized Sharpe for daily rebalancing.}
  \label{fig:ic-vs-sharpe-reb1}
\end{figure}

\begin{figure}[H]
  \centering
  \includegraphics[width=\textwidth]{ic_vs_sharpe_reb5.png}
  \caption{Rank IC versus annualized Sharpe for weekly rebalancing.}
  \label{fig:ic-vs-sharpe-reb5}
\end{figure}

\begin{figure}[H]
  \centering
  \includegraphics[width=\textwidth]{risk_frontier_reb1.png}
  \caption{Risk frontier (annualized return versus max drawdown) for daily rebalancing.}
  \label{fig:risk-frontier-reb1}
\end{figure}

\begin{figure}[H]
  \centering
  \includegraphics[width=\textwidth]{risk_frontier_reb5.png}
  \caption{Risk frontier (annualized return versus max drawdown) for weekly rebalancing.}
  \label{fig:risk-frontier-reb5}
\end{figure}

\begin{figure}[H]
  \centering
  \includegraphics[width=\textwidth]{bar_metrics_reb1.png}
  \caption{Top-model metric bars for daily rebalancing.}
  \label{fig:bar-metrics-reb1}
\end{figure}

\begin{figure}[H]
  \centering
  \includegraphics[width=\textwidth]{bar_metrics_reb5.png}
  \caption{Top-model metric bars for weekly rebalancing.}
  \label{fig:bar-metrics-reb5}
\end{figure}

\section{Main Comparison Tables}
% Auto-generated by thesis/scripts/export_tables.py
\begin{table}[htbp]
\caption{Professor-facing main table with Buy and hold, Equal weight, and best learned models.}
\label{tab:professor-main-results}
\centering
\small
\setlength{\tabcolsep}{4pt}
\resizebox{\textwidth}{!}{%
\begin{tabular}{llllrrrrrr}
\toprule
Strategy & Policy & Type & Rebalance & Final & Ann Return & Ann Vol & Sharpe & Max DD & Turnover \\
\midrule
Buy and hold (fixed shares) & Buy and hold (fixed shares) & baseline & reb=1 & 1.5012 & 0.5061 & 0.2371 & 2.1340 & -0.1028 & 0.0000 \\
Buy and hold (fixed shares) & Buy and hold (fixed shares) & baseline & reb=5 & 1.5012 & 0.5061 & 0.2371 & 2.1340 & -0.1028 & 0.0000 \\
Buy and hold (fixed shares) & Buy and hold (fixed shares) & baseline & reb=21 & 1.9076 & 0.9225 & 0.4296 & 2.1473 & -0.1889 & 0.0000 \\
Equal weight (rebalanced, all assets) & Equal weight (rebalanced, all assets) & baseline & reb=1 & 1.3608 & 0.3642 & 0.1432 & 2.5425 & -0.0441 & 0.0110 \\
Equal weight (rebalanced, all assets) & Equal weight (rebalanced, all assets) & baseline & reb=5 & 1.3614 & 0.3648 & 0.1432 & 2.5479 & -0.0443 & 0.0057 \\
Equal weight (rebalanced, all assets) & Equal weight (rebalanced, all assets) & baseline & reb=21 & 1.2595 & 0.2630 & 0.1213 & 2.1678 & -0.0594 & 0.0034 \\
gat\_corr\_sector\_granger\_rolling\_cv\_y2021 (Top K long-only (equal weight within Top K)) & Top K long-only (equal weight within Top K) & model & reb=5 & 1.3376 & 0.3257 & 0.1229 & 2.4386 & -0.0578 & 0.1510 \\
xgb\_node2vec\_corr\_rolling\_cv\_y2021 (Top K long-only (equal weight within Top K)) & Top K long-only (equal weight within Top K) & model & reb=5 & 1.2906 & 0.2825 & 0.1151 & 2.2831 & -0.0559 & 0.1371 \\
xgb\_raw\_rolling\_cv\_y2021 (Top K long-only (equal weight within Top K)) & Top K long-only (equal weight within Top K) & model & reb=5 & 1.2918 & 0.2818 & 0.1178 & 2.2417 & -0.0490 & 0.1470 \\
\bottomrule
\end{tabular}
}
\end{table}

% Auto-generated by thesis/scripts/export_tables.py
\begin{table}[htbp]
\caption{Baseline benchmark metrics in the test window (Buy and hold fixed shares, Equal weight rebalanced all assets).}
\label{tab:baseline-policy-comparison}
\centering
\small
\setlength{\tabcolsep}{4pt}
\resizebox{\textwidth}{!}{%
\begin{tabular}{llrrrrll}
\toprule
Strategy & Reb & Sharpe & Ann Return & Max DD & Final & Test Start & Test End \\
\midrule
Buy and hold (fixed shares) & reb=1 & 1.2273 & 0.3842 & -0.3471 & 5.0493 & 2020-01-02 & 2024-12-27 \\
Equal weight (rebalanced, all assets) & reb=1 & 0.7667 & 0.1519 & -0.3465 & 2.0222 & 2020-01-02 & 2024-12-27 \\
Buy and hold (fixed shares) & reb=5 & 1.2273 & 0.3842 & -0.3471 & 5.0493 & 2020-01-02 & 2024-12-27 \\
Equal weight (rebalanced, all assets) & reb=5 & 0.7667 & 0.1519 & -0.3465 & 2.0222 & 2020-01-02 & 2024-12-27 \\
\bottomrule
\end{tabular}
}
\end{table}

% Auto-generated by thesis/scripts/export_tables.py
\begin{table}[htbp]
\caption{Top models by Sharpe (rebalance\_freq=1).}
\label{tab:top-models-reb1}
\centering
\small
\setlength{\tabcolsep}{4pt}
\resizebox{\textwidth}{!}{%
\begin{tabular}{llrrrr}
\toprule
Model & Family & Rank IC & Sharpe & Max DD & Final \\
\midrule
XGB+Node2Vec (n2v-corr) & XGB & 0.0013 & 2.4224 & -0.0623 & 1.3176 \\
GAT (corr+sec+gr) & GNN & -0.0125 & 2.3337 & -0.0567 & 1.3252 \\
XGB & XGB & 0.0010 & 2.3078 & -0.0448 & 1.3068 \\
XGB+Node2Vec (n2v-corr) & XGB & 0.0127 & 2.2857 & -0.0483 & 1.2610 \\
GAT (corr+sec+gr) & GNN & 0.0060 & 2.1880 & -0.0584 & 1.2801 \\
XGB & XGB & -0.0026 & 1.8310 & -0.0530 & 1.2030 \\
GAT (corr+sec+gr) & GNN & 0.0026 & 1.7897 & -0.0958 & 1.2348 \\
XGB+Node2Vec (n2v-corr) & XGB & 0.0136 & 1.6539 & -0.0975 & 1.2134 \\
\bottomrule
\end{tabular}
}
\end{table}

% Auto-generated by thesis/scripts/export_tables.py
\begin{table}[htbp]
\caption{Top models by Sharpe (rebalance\_freq=5).}
\label{tab:top-models-reb5}
\centering
\small
\setlength{\tabcolsep}{4pt}
\resizebox{\textwidth}{!}{%
\begin{tabular}{llrrrr}
\toprule
Model & Family & Rank IC & Sharpe & Max DD & Final \\
\midrule
XGB+Node2Vec (n2v-corr) & XGB & 0.0160 & 0.8058 & -0.3284 & 2.0112 \\
MLP & mlp & -0.0029 & 0.7983 & -0.3305 & 2.0996 \\
XGB & XGB & 0.0043 & 0.7390 & -0.3117 & 1.8726 \\
LSTM & LSTM & -0.0010 & 0.7143 & -0.3473 & 1.9008 \\
GAT (corr+sec+gr) & GNN & -0.0077 & 0.7030 & -0.3089 & 1.8531 \\
TGAT-static (corr) & GNN & 0.0037 & 0.6923 & -0.3455 & 1.9014 \\
GCN (sector) & GNN & 0.0058 & 0.6916 & -0.3402 & 1.8620 \\
GAT (sector) & GNN & 0.0035 & 0.6696 & -0.3184 & 1.8176 \\
\bottomrule
\end{tabular}
}
\end{table}

% Auto-generated by thesis/scripts/export_tables.py
\begin{table}[htbp]
\caption{Top model runs and baseline benchmarks in one view (rebalance\_freq=1).}
\label{tab:model-vs-baseline-reb1}
\centering
\small
\setlength{\tabcolsep}{4pt}
\resizebox{\textwidth}{!}{%
\begin{tabular}{lllrrrr}
\toprule
Strategy & Type & Reb & Sharpe & Ann Return & Max DD & Final \\
\midrule
mlp & Model & reb=1 & 0.8703 & 0.1749 & -0.2806 & 2.2201 \\
XGB+Node2Vec (corr) & Model & reb=1 & 0.6602 & 0.1200 & -0.2948 & 1.7477 \\
GCN (sector) & Model & reb=1 & 0.6061 & 0.1126 & -0.3194 & 1.6848 \\
GAT (corr) & Model & reb=1 & 0.5625 & 0.1055 & -0.3511 & 1.6311 \\
GCN (granger) & Model & reb=1 & 0.5541 & 0.0954 & -0.2902 & 1.5644 \\
GCN (corr+sec+gr) & Model & reb=1 & 0.5365 & 0.0895 & -0.2511 & 1.5212 \\
LSTM & Model & reb=1 & 0.5323 & 0.0975 & -0.3523 & 1.5743 \\
XGB & Model & reb=1 & 0.5115 & 0.0854 & -0.2915 & 1.4995 \\
Buy and hold (fixed shares) & Baseline & reb=1 & 1.2273 & 0.3842 & -0.3471 & 5.0493 \\
Equal weight (rebalanced, all assets) & Baseline & reb=1 & 0.7667 & 0.1519 & -0.3465 & 2.0222 \\
\bottomrule
\end{tabular}
}
\end{table}

% Auto-generated by thesis/scripts/export_tables.py
\begin{table}[htbp]
\caption{Top model runs and baseline benchmarks in one view (rebalance\_freq=5).}
\label{tab:model-vs-baseline-reb5}
\centering
\small
\setlength{\tabcolsep}{4pt}
\resizebox{\textwidth}{!}{%
\begin{tabular}{lllrrrr}
\toprule
Strategy & Type & Reb & Sharpe & Ann Return & Max DD & Final \\
\midrule
XGB+Node2Vec (corr) & Model & reb=5 & 0.8058 & 0.1519 & -0.3284 & 2.0112 \\
mlp & Model & reb=5 & 0.7983 & 0.1618 & -0.3305 & 2.0996 \\
XGB & Model & reb=5 & 0.7390 & 0.1349 & -0.3117 & 1.8726 \\
LSTM & Model & reb=5 & 0.7143 & 0.1398 & -0.3473 & 1.9008 \\
GAT (corr+sec+gr) & Model & reb=5 & 0.7030 & 0.1334 & -0.3089 & 1.8531 \\
TGAT-static (corr) & Model & reb=5 & 0.6923 & 0.1388 & -0.3455 & 1.9014 \\
GCN (sector) & Model & reb=5 & 0.6916 & 0.1351 & -0.3402 & 1.8620 \\
GAT (sector) & Model & reb=5 & 0.6696 & 0.1296 & -0.3184 & 1.8176 \\
Buy and hold (fixed shares) & Baseline & reb=5 & 1.2273 & 0.3842 & -0.3471 & 5.0493 \\
Equal weight (rebalanced, all assets) & Baseline & reb=5 & 0.7667 & 0.1519 & -0.3465 & 2.0222 \\
\bottomrule
\end{tabular}
}
\end{table}

% Auto-generated by thesis/scripts/export_tables.py
\begin{table}[htbp]
\caption{Best runs by model category and rebalance policy.}
\label{tab:family-summary}
\centering
\small
\setlength{\tabcolsep}{4pt}
\resizebox{\textwidth}{!}{%
\begin{tabular}{lllrrr}
\toprule
Category & Rebalance & Model & Rank IC & Sharpe & Final \\
\midrule
Graph-feature & reb=1 & XGB+Node2Vec (corr) & 0.0160 & 0.6602 & 1.7477 \\
Non-graph & reb=1 & LSTM & -0.0010 & 0.5323 & 1.5743 \\
Static GNN & reb=1 & GCN (sector) & 0.0058 & 0.6061 & 1.6848 \\
Static temporal & reb=1 & TGAT-static (corr) & 0.0037 & 0.5055 & 1.5485 \\
Graph-feature & reb=5 & XGB+Node2Vec (corr) & 0.0160 & 0.8058 & 2.0112 \\
Non-graph & reb=5 & XGB & 0.0043 & 0.7390 & 1.8726 \\
Static GNN & reb=5 & GAT (corr+sec+gr) & -0.0077 & 0.7030 & 1.8531 \\
Static temporal & reb=5 & TGAT-static (corr) & 0.0037 & 0.6923 & 1.9014 \\
\bottomrule
\end{tabular}
}
\end{table}

% Auto-generated by thesis/scripts/export_tables.py
\begin{table}[htbp]
\caption{Edge-type ablation summary from thesis report output.}
\label{tab:edge-ablation}
\centering
\small
\setlength{\tabcolsep}{4pt}
\resizebox{\textwidth}{!}{%
\begin{tabular}{rlrrrr}
\toprule
Rebalance & Edge & N & Mean IC & Mean Sharpe & Mean MDD \\
\midrule
1 & corr & 2 & -0.0012 & NaN & -0.3153 \\
1 & corr+sec+gr & 2 & 0.0057 & NaN & -0.3036 \\
1 & granger & 2 & 0.0080 & NaN & -0.3073 \\
1 & sector & 2 & 0.0066 & NaN & -0.3255 \\
5 & corr & 2 & -0.0012 & NaN & -0.3404 \\
5 & corr+sec+gr & 2 & 0.0057 & NaN & -0.3339 \\
5 & granger & 2 & 0.0080 & NaN & -0.3337 \\
5 & sector & 2 & 0.0066 & NaN & -0.3349 \\
\bottomrule
\end{tabular}
}
\end{table}

% Auto-generated by thesis/scripts/export_tables.py
\begin{table}[htbp]
\caption{Fail-fast audit status for equal-weight rebalance integrity and graph time-awareness.}
\label{tab:audit-status}
\centering
\small
\setlength{\tabcolsep}{4pt}
\resizebox{\textwidth}{!}{%
\begin{tabular}{llrrl}
\toprule
Audit & Status & Fails & Warnings & Detail \\
\midrule
equal\_weight\_rebalance\_integrity & pass & 0 & 0 & fail if reb=1 and reb=5 EQW series are identical \\
graph\_time\_awareness & pass & 0 & 0 & fail if graph max timestamp exceeds policy bound \\
\bottomrule
\end{tabular}
}
\end{table}

% Auto-generated by thesis/scripts/export_tables.py
\begin{table}[htbp]
\caption{Active-vs-equal-weight diagnostics (arithmetic active returns, annualized with 252 trading days).}
\label{tab:alpha-vs-equal-weight}
\centering
\small
\setlength{\tabcolsep}{4pt}
\resizebox{\textwidth}{!}{%
\begin{tabular}{llrrrrr}
\toprule
Run & Reb & Active Return & Tracking Error & IR & Rel Final & Rel Max DD \\
\midrule
mlp & reb=1 & 0.0261 & 0.3095 & 0.0844 & 1.1174 & -0.1760 \\
XGB+Node2Vec (corr) & reb=1 & -0.0236 & 0.3043 & -0.0775 & 0.8796 & -0.2312 \\
GCN (sector) & reb=1 & -0.0284 & 0.3110 & -0.0914 & 0.8480 & -0.2532 \\
GAT (corr) & reb=1 & -0.0332 & 0.3182 & -0.1042 & 0.8209 & -0.2511 \\
LSTM & reb=1 & -0.0408 & 0.3157 & -0.1293 & 0.7924 & -0.2674 \\
TGAT-static (corr) & reb=1 & -0.0433 & 0.3210 & -0.1349 & 0.7794 & -0.3143 \\
GAT (granger) & reb=1 & -0.0485 & 0.3148 & -0.1539 & 0.7645 & -0.3014 \\
GCN (granger) & reb=1 & -0.0464 & 0.3009 & -0.1541 & 0.7874 & -0.3123 \\
mlp & reb=5 & 0.0184 & 0.3168 & 0.0582 & 1.0661 & -0.2042 \\
XGB+Node2Vec (corr) & reb=5 & 0.0064 & 0.3032 & 0.0210 & 1.0211 & -0.1946 \\
TGAT-static (corr) & reb=5 & -0.0003 & 0.3210 & -0.0008 & 0.9654 & -0.1975 \\
LSTM & reb=5 & -0.0016 & 0.3115 & -0.0053 & 0.9651 & -0.1858 \\
GCN (sector) & reb=5 & -0.0053 & 0.3160 & -0.0169 & 0.9454 & -0.2020 \\
GAT (corr+sec+gr) & reb=5 & -0.0084 & 0.3078 & -0.0273 & 0.9409 & -0.1939 \\
XGB & reb=5 & -0.0089 & 0.3034 & -0.0295 & 0.9508 & -0.1949 \\
GAT (sector) & reb=5 & -0.0103 & 0.3156 & -0.0326 & 0.9229 & -0.2038 \\
\bottomrule
\end{tabular}
}
\end{table}

% Auto-generated by thesis/scripts/export_tables.py
\begin{table}[htbp]
\caption{Market-neutral long-short results (top 3 long, bottom 3 short; gross, cost bps = 0).}
\label{tab:long-short-top3-bottom3}
\centering
\small
\setlength{\tabcolsep}{4pt}
\resizebox{\textwidth}{!}{%
\begin{tabular}{llrrrrrrrrr}
\toprule
Run & Reb & Sharpe & Ann Return & Max DD & Final & Turnover & Long K & Short K & Long Sum & Short Sum \\
\midrule
XGB & reb=5 & 0.6212 & 0.0759 & -0.2224 & 1.4479 & 0.3237 & 3 & 3 & 0.5000 & 0.5000 \\
XGB+Node2Vec (corr) & reb=5 & 0.5283 & 0.0584 & -0.2048 & 1.3210 & 0.3101 & 3 & 3 & 0.5000 & 0.5000 \\
TGAT-static (corr) & reb=5 & 0.4559 & 0.0630 & -0.1848 & 1.3673 & 0.3025 & 3 & 3 & 0.5000 & 0.5000 \\
mlp & reb=5 & 0.3925 & 0.0515 & -0.3864 & 1.2820 & 0.3246 & 3 & 3 & 0.5000 & 0.5000 \\
GAT (granger) & reb=5 & 0.3085 & 0.0362 & -0.1277 & 1.1830 & 0.3308 & 3 & 3 & 0.5000 & 0.5000 \\
GAT (sector) & reb=5 & 0.2625 & 0.0302 & -0.2297 & 1.1524 & 0.3445 & 3 & 3 & 0.5000 & 0.5000 \\
LSTM & reb=5 & 0.1844 & 0.0185 & -0.3184 & 1.0855 & 0.3198 & 3 & 3 & 0.5000 & 0.5000 \\
GAT (corr) & reb=5 & 0.1464 & 0.0120 & -0.1799 & 1.0548 & 0.3319 & 3 & 3 & 0.5000 & 0.5000 \\
GCN (corr+sec+gr) & reb=5 & 0.1261 & 0.0086 & -0.2088 & 1.0387 & 0.3298 & 3 & 3 & 0.5000 & 0.5000 \\
TGCN-static (corr) & reb=5 & 0.0091 & -0.0091 & -0.3738 & 0.9529 & 0.3465 & 3 & 3 & 0.5000 & 0.5000 \\
GCN (granger) & reb=5 & -0.1368 & -0.0293 & -0.2991 & 0.8635 & 0.3196 & 3 & 3 & 0.5000 & 0.5000 \\
GCN (sector) & reb=5 & -0.1507 & -0.0325 & -0.2758 & 0.8448 & 0.3355 & 3 & 3 & 0.5000 & 0.5000 \\
\bottomrule
\end{tabular}
}
\end{table}

% Auto-generated by thesis/scripts/export_tables.py
\begin{table}[htbp]
\caption{Cost sensitivity summary for long-only and long-short strategies (0/5/10 bps).}
\label{tab:cost-sensitivity-summary}
\centering
\small
\setlength{\tabcolsep}{4pt}
\resizebox{\textwidth}{!}{%
\begin{tabular}{lrlrrrrr}
\toprule
Strategy & Cost (bps) & Label & Mean Sharpe & Mean Ann Return & Mean Max DD & Mean Turnover & N \\
\midrule
long\_only\_topk & 0 & gross (0 bps) & 0.7971 & 0.1564 & -0.3166 & 0.3362 & 30 \\
long\_only\_topk & 5 & net (5 bps) & 0.5916 & 0.1086 & -0.3208 & 0.3362 & 30 \\
long\_only\_topk & 10 & net (10 bps) & 0.3862 & 0.0635 & -0.3284 & 0.3362 & 30 \\
long\_short\_top3\_bottom3 & 0 & gross (0 bps) & 0.1310 & 0.0097 & -0.2847 & 0.2053 & 30 \\
long\_short\_top3\_bottom3 & 5 & net (5 bps) & -0.0475 & -0.0159 & -0.3287 & 0.2053 & 30 \\
long\_short\_top3\_bottom3 & 10 & net (10 bps) & -0.2256 & -0.0405 & -0.3785 & 0.2053 & 30 \\
\bottomrule
\end{tabular}
}
\end{table}

% Auto-generated by thesis/scripts/export_tables.py
\begin{table}[htbp]
\caption{Subset comparison for monthly rebalancing extension (rebalance\_freq=21) with gross costs.}
\label{tab:monthly-rebalance-subset}
\centering
\small
\setlength{\tabcolsep}{4pt}
\resizebox{\textwidth}{!}{%
\begin{tabular}{llllrrrrr}
\toprule
Strategy & Kind & Reb & Cost & Sharpe & Ann Return & Max DD & Final & Turnover \\
\midrule
xgb\_node2vec\_corr\_tuned\_all & model\_topk\_long\_only & reb=1 & gross (0 bps) & 1.0167 & 0.2025 & -0.2894 & 2.4928 & 0.5657 \\
xgb\_node2vec\_corr\_tuned\_all & model\_topk\_long\_only & reb=5 & gross (0 bps) & 0.8919 & 0.1717 & -0.3275 & 2.1904 & 0.1361 \\
xgb\_node2vec\_corr\_tuned\_all & model\_topk\_long\_only & reb=21 & gross (0 bps) & 0.7220 & 0.1309 & -0.2969 & 1.8353 & 0.0342 \\
xgb\_raw\_tuned\_all & model\_topk\_long\_only & reb=1 & gross (0 bps) & 0.8974 & 0.1709 & -0.2850 & 2.1886 & 0.6022 \\
xgb\_raw\_tuned\_all & model\_topk\_long\_only & reb=5 & gross (0 bps) & 0.8285 & 0.1550 & -0.3103 & 2.0447 & 0.1402 \\
xgb\_raw\_tuned\_all & model\_topk\_long\_only & reb=21 & gross (0 bps) & 0.7784 & 0.1359 & -0.2744 & 1.8816 & 0.0359 \\
gat\_corr\_sector\_granger\_tuned\_all & model\_topk\_long\_only & reb=1 & gross (0 bps) & 0.8089 & 0.1590 & -0.3158 & 2.0726 & 0.5822 \\
gat\_corr\_sector\_granger\_tuned\_all & model\_topk\_long\_only & reb=5 & gross (0 bps) & 0.7924 & 0.1544 & -0.3079 & 2.0320 & 0.1469 \\
gat\_corr\_sector\_granger\_tuned\_all & model\_topk\_long\_only & reb=21 & gross (0 bps) & 0.7653 & 0.1451 & -0.2984 & 1.9519 & 0.0372 \\
Equal weight (rebalanced, all assets) & baseline\_equal\_weight & reb=1 & gross (0 bps) & 0.8042 & 0.1477 & -0.3112 & 1.9869 & 0.0115 \\
Buy and hold (fixed shares) & baseline\_buy\_and\_hold & reb=1 & gross (0 bps) & 0.9566 & 0.1929 & -0.2986 & 2.4074 & 0.0000 \\
Equal weight (rebalanced, all assets) & baseline\_equal\_weight & reb=5 & gross (0 bps) & 0.7976 & 0.1457 & -0.3136 & 1.9695 & 0.0057 \\
Buy and hold (fixed shares) & baseline\_buy\_and\_hold & reb=5 & gross (0 bps) & 0.9566 & 0.1929 & -0.2986 & 2.4074 & 0.0000 \\
Equal weight (rebalanced, all assets) & baseline\_equal\_weight & reb=21 & gross (0 bps) & 0.8033 & 0.1456 & -0.3096 & 1.9685 & 0.0033 \\
Buy and hold (fixed shares) & baseline\_buy\_and\_hold & reb=21 & gross (0 bps) & 0.9566 & 0.1929 & -0.2986 & 2.4074 & 0.0000 \\
\bottomrule
\end{tabular}
}
\end{table}

% Auto-generated by thesis/scripts/export_tables.py
\begin{table}[htbp]
\caption{Lookback sensitivity subset (14/30/60); missing entries require retraining.}
\label{tab:lookback-sensitivity-subset}
\centering
\small
\setlength{\tabcolsep}{4pt}
\resizebox{\textwidth}{!}{%
\begin{tabular}{llllrlrrrr}
\toprule
Run & Model & Edge & Reb & Lookback & Status & Sharpe & Ann Return & Max DD & Final \\
\midrule
XGB+Node2Vec (corr) & xgb\_node2vec & node2vec\_correlation & reb=5 & 14 & missing\_rerun\_required & NaN & NaN & NaN & NaN \\
XGB+Node2Vec (corr) & xgb\_node2vec & node2vec\_correlation & reb=5 & 30 & missing\_rerun\_required & NaN & NaN & NaN & NaN \\
XGB+Node2Vec (corr) & xgb\_node2vec & node2vec\_correlation & reb=5 & 60 & available & 2.2831 & 0.2825 & -0.0559 & 1.2906 \\
XGB & xgb\_raw & none & reb=5 & 14 & missing\_rerun\_required & NaN & NaN & NaN & NaN \\
XGB & xgb\_raw & none & reb=5 & 30 & missing\_rerun\_required & NaN & NaN & NaN & NaN \\
XGB & xgb\_raw & none & reb=5 & 60 & available & 2.2417 & 0.2818 & -0.0490 & 1.2918 \\
GAT (corr+sec+gr) & gat & corr+sector+granger & reb=5 & 14 & missing\_rerun\_required & NaN & NaN & NaN & NaN \\
GAT (corr+sec+gr) & gat & corr+sector+granger & reb=5 & 30 & missing\_rerun\_required & NaN & NaN & NaN & NaN \\
GAT (corr+sec+gr) & gat & corr+sector+granger & reb=5 & 60 & missing\_rerun\_required & NaN & NaN & NaN & NaN \\
\bottomrule
\end{tabular}
}
\end{table}


\section{Alpha and Market-Neutral Diagnostics}
\begin{figure}[H]
  \centering
  \includegraphics[width=\textwidth]{relative_wealth_key_models.png}
  \caption{Relative wealth of model portfolios vs Equal weight (rebalanced, all assets).}
  \label{fig:relative-wealth-vs-eqw}
\end{figure}

\begin{figure}[H]
  \centering
  \includegraphics[width=\textwidth]{active_ir_vs_active_return.png}
  \caption{Active annualized return vs information ratio relative to Equal weight (rebalanced, all assets).}
  \label{fig:active-ir-vs-return}
\end{figure}

\begin{figure}[H]
  \centering
  \includegraphics[width=\textwidth]{long_short_equity_top3_bottom3.png}
  \caption{Market-neutral long-short equity curves (top 3 long, bottom 3 short; gross).}
  \label{fig:long-short-top3-bottom3}
\end{figure}

\begin{figure}[H]
  \centering
  \includegraphics[width=\textwidth]{cost_sensitivity_summary.png}
  \caption{Cost sensitivity summary under 0/5/10 bps.}
  \label{fig:cost-sensitivity-summary}
\end{figure}

\section{Primary Findings}
\textbf{1) Weekly rebalancing is systematically better in this benchmark.}
Across the canonical model variants, moving from daily to weekly rebalancing improves annualized Sharpe and reduces turnover for the majority of runs. This is consistent with lower implementation friction from less frequent trading.

\textbf{2) The top overall run is the graph-feature baseline.}
\texttt{xgb\_node2vec\_corr\_tuned\_all} is rank-1 under \texttt{rebalance\_freq=5}, with strong Sharpe and competitive drawdown. The result supports the claim that relational information is useful even when injected as features into a non-graph learner.

\textbf{3) Prediction metrics and portfolio metrics are related but not equivalent.}
RMSE/MAE spread is narrow relative to the spread in Rank IC and Sharpe. Several runs with similar point error produce materially different turnover and drawdown, showing why portfolio-aware evaluation is required.

\textbf{4) Family-level averages do not imply universal model dominance.}
XGBoost-family variants lead on mean Sharpe in this window, while GNNs remain competitive in selected edge settings. This supports a conditional conclusion: representation choice matters, but deployment utility depends on the full stack (edge design, rebalance policy, and costs).

\textbf{5) Edge ablation favors simpler sector priors over merged complexity on average.}
Sector-edge variants are strongest on mean GNN Sharpe in this tuned matrix. The combined \texttt{corr+sector+granger} graph does not dominate average outcomes, indicating that adding channels can introduce noise or instability.

\section{Buy-and-Hold Gap and Alpha Claims}
A critical result is that all tested model-driven strategies remain below test-window buy-and-hold final wealth in this period. Therefore, this thesis does \emph{not} make a broad alpha-superiority claim. The supported claim is narrower: under one controlled protocol, some graph-aware variants improve risk-adjusted portfolio diagnostics relative to other learned models, but they do not beat passive benchmark wealth in this regime.

\section{Research-Question Mapping}
\textbf{RQ1 (graph-based predictive quality):} partially supported. Graph-aware information helps, especially in graph-feature hybrids, but end-to-end GNNs are not uniformly dominant.

\textbf{RQ2 (useful edge signals):} sector edges are the most robust average GNN choice in this matrix.

\textbf{RQ3 (monetization of prediction gains):} partially supported. Improvements in ranking and Sharpe exist within model families, but absolute wealth remains below buy-and-hold in the 2020--2024 window.

\textbf{RQ4 (rebalance sensitivity):} strongly supported. Weekly rebalancing materially improves deployable quality by reducing turnover and often improving Sharpe.
